\documentclass[12pt, letterpaper]{article}

\usepackage{graphicx}
\usepackage{listings}

\renewcommand\maketitle{
    \begin{flushleft}
        Name: Benjamin Kammer \\
        MatrNr: 205538
    \end{flushleft}

    \begin{flushright}\vspace{-15mm}
        \includegraphics[width=3.1cm]{logo.png}
    \end{flushright}

    \begin{center}
        \textbf{\large Studium Generale: Escape-Room}\\
        Dokumentation: Chemieroom
    \end{center}

    \rule{\linewidth}{0.1mm}

    \bigskip
}

\begin{document}

\maketitle

\section{Aufsetzen des digitalen Escape-Rooms}
\subsection{Komponenten}

\begin{itemize}
	\item Raspberry Pi 400
    \item Mikro-SD Karte
	\item Computer Maus
	\item Mikro-HDMI auf HDMI Kabel
	\item Monitor
    \item Computer mit SD-Kartensteckplatz
\end{itemize}

\subsection{Präparation des Raspberry Pis}

Installiere den Raspberry Pi Imager\footnote{https://www.raspberrypi.com/software/} auf einem Computer.
Stecke die SD-Karte in den Computer und öffne den Raspberry Pi Imager.
Wähle unter "Operating System" das "Raspberry Pi OS (64-Bit)" und unter "Storage" die SD-Karte aus.
Klicke anschließend auf "write".
Ein Tutorial\footnote{https://www.youtube.com/watch?v=ntaXWS8Lk34} von der Raspberry Pi Foundation für die Installation findet sich auf YouTube.

Stecke nach einem erfolgreichen Schreibvorgang die SD-Karte in den Raspberry Pi 400 ein.
Verbinde ebenfalls die Maus und den Monitor mit dem HDMI-Kabel mit dem Raspberry Pi und verbinde diesen anschließend mit Strom.
Befolge die folgenden Einrichtungsschritte auf dem Monitor.

Verbinde dich mit dem Internet, falls dies noch nicht geschehen ist.
Öffne das Programm Terminal und führe den folgenden Befehl aus:
\begin{lstlisting}[language=bash]
curl -sSL
https://raw.githubusercontent.com/bkammer/
escape-room/main/pi-install.sh
| bash
\end{lstlisting}

Öffne anschließend mit dem Browser des Raspberry Pis die URL: escaperoom.de.
Diese Adresse ist nur auf dem Raspberry Pi bekannt und zeigt auf den Webserver, der im Hintergrund auf dem Raspberry Pi läuft.
Damit ist zukünftig keine Internetverbindung mehr notwendig.

\section{Durchführung}
\subsection{Komponenten}

\begin{itemize}
	\item Raspberry Pi 400
    \item Mikro-HDMI Karte
	\item Computer Maus
	\item Mikro-HDMI auf HDMI Kabel
	\item Monitor
\end{itemize}

\subsection{Ausgangszustand}

Für die Durchführung als Rätsel im Escape-Room sind diese Teile bereitzustellen.
Die Komponenten „Herstellung“ wird nur für die Präparation des Raspberry Pis benötigt, welches vorher stattfinden muss.

\subsection{Spielprinzip}

Auf der Startseite befindet sich ein Startknopf, auf dessen Knopfdruck ein Timer anfängt zu laufen.
Das Spielprinzip ist es, in dem gefangenen Haus möglichst viele Räume zu lösen, um dabei an Geld zu gelangen, mit dem man sich aus dem Haus freikaufen kann.

\subsection{Erklärung}

\subsection{Anwendung}

\section{Weiterführende Links und Literatur}

\section{Rätselfrage}

\end{document}
