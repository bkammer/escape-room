\documentclass[12pt, letterpaper]{article}

\usepackage{graphicx}
\usepackage{listings}
\usepackage[margin=2.54cm]{geometry}

\graphicspath{ {./images/} }
\renewcommand{\figurename}{Grafik}

\renewcommand\maketitle{
    \begin{flushleft}
        Name: Benjamin Kammer \\
        MatrNr: 205538
    \end{flushleft}

    \begin{flushright}\vspace{-15mm}
        \includegraphics[width=3.1cm]{logo.png}
    \end{flushright}

    \begin{center}
        \textbf{\large Studium Generale: Escape-Room}\\
        Dokumentation: Informatik-Raum-2
    \end{center}

    \rule{\linewidth}{0.1mm}

    \bigskip
}

\begin{document}

\maketitle

\section{Digitalen Escape-Room aufsetzen}
\subsection{Komponenten}

\begin{itemize}
	\item Raspberry Pi 400
    \item Mikro-SD Karte
	\item Computer Maus
	\item Mikro-HDMI auf HDMI Kabel
	\item Monitor / Bildschirm
    \item Computer mit SD-Kartensteckplatz
\end{itemize}

\subsection{Präparation des Raspberry Pis}

Installiere den Raspberry Pi Imager\footnote{https://www.raspberrypi.com/software/} auf einem Computer.
Stecke die SD-Karte in den Computer und öffne den Raspberry Pi Imager.
Wähle unter "Operating System" das "Raspberry Pi OS (64-Bit)" und unter "Storage" die eingesteckte SD-Karte aus.
Klicke anschließend auf "write".
Ein Tutorial\footnote{https://www.youtube.com/watch?v=ntaXWS8Lk34} von der Raspberry Pi Foundation für die Installation findet sich auf YouTube.

Stecke nach einem erfolgreichen Schreibvorgang die SD-Karte in den Raspberry Pi 400 ein.
Verbinde ebenfalls die Maus und den Monitor mit dem HDMI-Kabel mit dem Raspberry Pi und verbinde diesen anschließend mit Strom.
Befolge die folgenden Einrichtungsschritte auf dem Bildschirm.
Verbinde dich mit dem Internet, falls dies noch nicht geschehen ist.
Öffne das Programm Terminal und führe folgenden Befehl aus:

\begin{verbatim}
curl -sSL https://raw.githubusercontent.com/bkammer/escape-room/main/pi-install.sh | bash
\end{verbatim}

Öffne anschließend mit dem Browser des Raspberry Pis die URL: escaperoom.de.
Diese Adresse ist nur auf dem Raspberry Pi bekannt und zeigt auf den Webserver, der im Hintergrund auf dem Raspberry Pi läuft.
Damit ist zukünftig keine Internetverbindung mehr notwendig, um den digitalen Escaperoom zu spielen.

\section{Durchführung}
\subsection{Komponenten}

\begin{itemize}
	\item Raspberry Pi 400
    \item Mikro-HDMI Karte
	\item Computer Maus
	\item Mikro-HDMI auf HDMI Kabel
	\item Monitor
\end{itemize}

\subsection{Ausgangszustand}

Für die Durchführung als Rätsel im Escape-Room sind diese Teile bereitzustellen.
Die Komponenten im Abschnitt Herstellung wird nur für die Präparation des Raspberry Pis benötigt, welche vorher stattfinden muss.

\subsection{Spielprinzip}

Durch Betätigen des Start-Knopfes auf der Startseite beginnt ein Countdown nach unten zu zählen.
Innerhalb dieser Zeit müssen verschiedene Räume gelöst werden, um Münzen zu verdienen.
Besitzt man genügend Münzen, so kann man sich aus dem Escape-Haus freikaufen.
Schafft man es nicht, bis zum Ablaufen des Countdowns genügend Münzen zu sammeln und sich freizukaufen, so ist man gescheitert.

\subsection{Informatikraum 2}



\subsection{Erklärung}

Bei der Caesar-Verschlüsselung, auch Verschiebe-Chiffre genannt, werden Texte verschlüsselt, indem man die Buchstaben des geordneten Alphabets um eine bestimmte Anzahl zyklisch verschiebt.
Eine schematische Darstellung mit der Verschiebung um 3 Buchstaben ist in Grafik \ref{fig:caesar3} dargestellt.

\begin{figure}[h]
    \centering
    \includegraphics[width=0.5\textwidth]{caesar3}
    \caption{Schematische Darstellung einer Verschiebe-Chiffre mit Verschiebung um drei Buchstaben (https://de.wikipedia.org/wiki/Caesar-Verschlüsselung)}
    \label{fig:caesar3}
\end{figure}

\subsection{Anwendung}

Die Caesar-Verschlüsselung zählt zu einer der unsichersten Verfahren der heutigen Kryptografie.
Denn diese kann durch eine Häufigkeitsanalyse entschlüsselt werden.
Dabei werden die einzelnen Buchstaben des verschlüsselten Textes gezählt und ihre Häufigkeit, meist in Prozent, notiert.
In der deutschen Sprache kommt beispielsweise das E mit ungefähr 17,4 Prozent am häufigsten vor.
Eine Liste mit der Buchstabenhäufigkeit der deutschen Sprache lässt sich auf Wikipedia\footnote{https://de.wikipedia.org/wiki/Buchstabenhäufigkeit} finden.
Somit kann auf das verwendete Alphabet geschlossen werden.
Die Genauigkeit der Häufigkeit steigt mit der Länge des Textes, womit ein langer Text leichter zu entschlüsseln ist als eine kurze.
Damit dient die Caesar-Verschlüsslung ausschließlich dafür, Grundprinzipien der Kryptografie näherzubringen.

\section{Weiterführende Links und Literatur}

\begin{itemize}
	\item https://de.wikipedia.org/wiki/Caesar-Verschlüsselung
    \item https://av.tib.eu/media/19812
    \item https://de.wikipedia.org/wiki/Häufigkeitsanalyse
    \item https://de.wikipedia.org/wiki/Buchstabenhöufigkeit
\end{itemize}

\section{Rätselfrage}

\end{document}
